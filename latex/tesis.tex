\documentclass[11pt,a4paper,twoside]{tesis}
% SI NO PENSAS IMPRIMIRLO EN FORMATO LIBRO PODES USAR
%\documentclass[11pt,a4paper]{tesis}

\usepackage{graphicx}
\usepackage{float}
\usepackage{titlesec}

\usepackage[dvipsnames]{xcolor}
\usepackage{listings}
\lstloadlanguages{Ruby}
\lstset{%
basicstyle=\ttfamily\color{black},
commentstyle = \ttfamily\color{red},
keywordstyle=\ttfamily\color{blue},
stringstyle=\color{orange}}

\definecolor{light-gray}{gray}{0.95}
\lstset{columns=fullflexible, basicstyle=\ttfamily,
    backgroundcolor=\color{light-gray},xleftmargin=1cm,frame=lr,framesep=8pt,framerule=0pt}


\usepackage[utf8]{inputenc}
\usepackage[spanish]{babel}
\usepackage[left=3cm,right=3cm,bottom=2cm,top=2cm]{geometry}
\setcounter{tocdepth}{5}
\setcounter{secnumdepth}{5}%Should do the work for the subsubsection, but it doesn't

\begin{document}

%%%% CARATULA
\def\titulo{Licenciado }

\def\autor{Andrés Laurito}
\def\tituloTesis{Haikunet: \mbox{SDN programming language}}
\def\runtitulo{Haikunet: a SDN programming language for debugging the network}
\def\runtitle{Haikunet: SDN programming language}
\def\director{Hernán Melgratti}
\def\codirector{Rodrigo Castro}
\def\lugar{Buenos Aires, 2017}
\input{caratula}

%%%% ABSTRACTS, AGRADECIMIENTOS Y DEDICATORIA
\frontmatter
\pagestyle{empty}
\input{abs_esp.tex}

\cleardoublepage
\input{abs_en.tex}

\cleardoublepage
\input{agradecimientos.tex}

\cleardoublepage
\input{dedicatoria.tex}

\cleardoublepage

\renewcommand*\contentsname{Summary}

\tableofcontents

\mainmatter
\pagestyle{headings}

%%%% ACA VA EL CONTENIDO DE LA TESIS

\chapter{Introduction}
\section{Motivations}
\begin{small}%
\begin{flushright}%
\it
The best way to predict the future, is to invent it. \\
--Alan Kay
\end{flushright}%
\end{small}%
\vspace{.5cm}

\chapter{Background}
\input{background.tex}

\chapter{TopologyGenerator}
\section{Base concept}

The topologygenerator is a tool for building a custom output file format out of a given network topology. The key concepts in this tool are the followings:
\begin{itemize}
\item Provider \\
A provider will be in charge of everything which is concerned to the network topology, meanning that it will have to provide information about it's elements, how they are connected between each other, general properties of each of the elements, etc., as well as has the ability to change it's content, for example adding new elements to the network, creating new connections, deleting elements, etc.\\
It's clear that in the eyes of the tool, the provider's abstraction will be the network and all the information concerned to it.

\item Builder \\


\item Output \\

\end{itemize}

\section{Tutorial}

The topology can be retrieved from a custom file written in ruby by the user, or from an SDN controller (by specifying the API uri). The ONOS controller is currently supported, while the API for OpenDayLight is in progress. When building your output, you have to write a module that describes how to each class defined in the network topology. The topologygenerator gem will then use the defined modules to generate the output desired. You can see examples of how to use this gem in the public github webpage.

\section{Architecture}

\section{Implementation}

\section{Limits}

\chapter{Haikunet}
\section{Conceptual Idea}



\section{Why Haikunet?}

The name Haikunet comes from the join of Haiku and Network. A Haiku is a Japanese poem with a lot of themes, but in most of the cases a Haiku describes the nature from the view of an observer. This is exactly what you are going to do with Haikunet, write short programs (almost poems), which will describe the desired nature of a network, and this description will be done from the point of view of an observer of the network, meaning a person which does not necessarily knows how to interact at a system level with it, and in most of the cases, doesn't know how to. This is the intent's programmer point of view of the network. 

\section{Tutorial}

\section{Formal Definition}

\section{Implementation}

\subsection{Lexer/Parser}

\subsection{Semantic Checker}

\subsection{Code Generation}

\section{Limitations}

\chapter{Conclusions}
\input{conclusions.tex}

%%%% BIBLIOGRAFIA
\backmatter
%\bibliography{tesis}

\end{document}
