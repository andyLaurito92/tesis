\documentclass[envcountsect,runningheads]{llncs}

%%%%%%%%%%%%%%%%%%%%%%%%%%%%%%%%
% ENLARGED STYLE FOR SUBMISSION
%%%%%%%%%%%%%%%%%%%%%%%%%%%%%%%%
%\setlength{\textwidth}{15cm}
%\setlength{\textheight}{21cm}
%\addtolength{\oddsidemargin}{-1.25cm}
%\setlength{\evensidemargin}{\oddsidemargin}
%%%%%%%%%%%%%%%%%%%%%%%%%%%%%%%%

\usepackage[all]{xy}\CompileMatrices
\usepackage[english]{babel}
\usepackage[mathcal]{euscript}
\usepackage{latexsym}
\usepackage{amssymb}
\usepackage{pslatex}
\usepackage{alltt}

\usepackage{float}
%\usepackage{natbib}
\usepackage{url}
\usepackage{subfigure}
\usepackage{stmaryrd}

%new
\newcommand{\red}{\rightarrow}
\newcommand{\rred}{\Rightarrow}
\newcommand{\jeq}{\; \mathop{\triangleright} \;}
\def \mathrule #1#2#3{\begin{array}{l}%
    {\mbox{\scriptsize ({\sc #1})} }%
    \\ \irule{#2}{#3}%
\end{array}}
\newcommand{\irule}[2]{\frac{\textstyle\rule[-1.3ex]{0cm}{3ex}#1}%
{\textstyle\rule[-.5ex]{0cm}{3ex}#2}}
\newcommand{\pntrans}{[\rangle}
\newcommand{\arule}[3]{\frac{\textstyle\rule[-.8ex]{0cm}{3ex}#1}%
{\textstyle\rule[.2ex]{0cm}{3ex}#2}{\mbox{\scriptsize {\sc #3}}}}



% concurrency relation
\newcommand{\co}[1]{\mathbf{co}(#1)}
%symbol for depth function
\newcommand{\depth}{\ensuremath{\mathit{depth}}}
%symbol for natural numbers
\newcommand{\nat}{\ensuremath{\mathbb{N}}}
% category of persistent graph grammars
\newcommand{\pGG}{\ensuremath{\mathbf{PGG}}}
% category of persistent occurrence graph grammars
\newcommand{\poGG}{\ensuremath{\mathbf{POGG}}}
% category of prime event structures
\newcommand{\pes}{\mathbf{PES}}


%
% macros from Paolo Baldan's files
%
\newcommand{\elem}[1]{\ensuremath{\mathit{Elem}(#1)}}
% partial maps
\newcommand{\pto}{\rightarrowtail}
% domain of a partial map
\newcommand{\dom}[1]{\ensuremath{\mathit{dom}(#1)}}
% category of graphs with partial morphisms
\newcommand{\PGraph}{\ensuremath{\mathbf{PGraph}}}
% category of graphs with total morphisms
\newcommand{\Graph}{\ensuremath{\mathbf{Graph}}}
% typed graphs: underlying graph
\newcommand{\tgr}[1]{\ensuremath{|{#1}|}}
% typed graphs: underlying morphism
\newcommand{\tmap}[1]{\ensuremath{\tau_{{#1}}}}
% category of T-typed graphs with partial morphisms
\newcommand{\PTyped}[1]{\ensuremath{{#1}\mbox{-}\mathbf{PGraph}}}
% category of T-typed graphs with total morphisms
\newcommand{\Typed}[1]{\ensuremath{{#1}\mbox{-}\mathbf{Graph}}}
% unfolding functor for SPO grammars
\newcommand{\Unfs}{\ensuremath{\mathcal{U}_s}}
% functor from AES to O-GG
\newcommand{\Ngg}{\ensuremath{{\mathcal{N}}_s}}
% symbol for semi-abstract span
\newcommand{\spar}{\ensuremath{\leftrightarrow}}
% empty production
\newcommand{\emptyprod}{\ensuremath{\emptyset}}
% xypic label
\newcommand{\lb}[3]{\save []+<#2, #3>*\txt{\ensuremath{#1}}
  \restore}
% preconditions
\newcommand{\pre}[1]{\mbox{${^\bullet}\!{#1}$}}
% postconditions
\newcommand{\post}[1]{\mbox{${{#1} {^\bullet}}$}}
% contexts
\newcommand{\cont}[1]{\ensuremath{\underline{#1}}}
% strong causes
\newcommand{\cause}[1]{\ensuremath{\lfloor {#1} \rfloor }}
% weak dependency relation
\newcommand{\wc}{\ensuremath{\nearrow}}

% domain of a partial map
\newcommand{\pmapincluded}{\subseteq}

%
% FIGURE SEPARATOR
%
%\newcommand{\topfigrule}{\vskip3pt\noindent\rule{\textwidth}{1pt}\vskip-15pt}

%
% MACRO FOR COMMENTS
%
%\newcommand{\nota}[1]{\noindent \fbox{ \parbox{\textwidth}{#1} }  }
\newcommand{\nota}[1]{}

%
% ABBREVIATIONS AND SYMBOLS
%
\newcommand{\eg}{e.g.}
\newcommand{\ie}{i.e.}
\newcommand{\wrt}{w.r.t.}




%
% JOIN PROCESSES
%
\newcommand{\tuple}[1]{\vec{#1}}
\newcommand{\zero} {0}
\newcommand{\mess}[2] { #1 \langle #2 \rangle}
\newcommand{\defproc}[2]{\textrm{\bf def } #1 \textrm{ \bf in } #2}
\newcommand{\basdef}[2]{ #1\triangleright  #2}
\newcommand{\mergingdef}[2]{ #1 \blacktriangleright #2}
\newcommand{\begintrans}[3]{ #1\triangleright  [#2:#3]}
\newcommand{\trans}[2]{[#1:#2]}
\newcommand{\abort}{\mathit{abort}}
\newcommand{\bigpar}{|\!|}

\newcommand{\membrane}[1]{\{\![ #1 ]\!\}}

\newcommand{\denote}[1]{\llbracket #1\rrbracket}

\newcommand{\outp}[2]{\overline{#1}\langle #2 \rangle}
\newcommand{\inp}[2]{#1(#2)}


%Macros Hernan
\newcommand{\join}{\textsf {Join}}
\newcommand{\fn}{\mathit{fn}}
\newcommand{\dn}{\mathit{dn}}
\newcommand{\rn}{\mathit{rn}}
\newcommand{\subst}[2]{\{^{#1}/_{#2}\}}
\newcommand{\airlock}[2]{\triangleright}
\newcommand{\frozen}[1]{\llcorner #1 \lrcorner}
\newcommand{\reduce}{\rightarrow}
\newtheorem{notation}{Notation}



\title{Nemo's graph grammar - The beggining of the end of Network's fireman approach}
\author{Andr\'es Laurito}

\institute{
Departamento de Computac\'on - FCEN -UBA\\
 \email{andy.laurito@gmail.com} }

\titlerunning{Nemo's graph grammar - The beggining of the end of Network's fireman approach}

\authorrunning{A. Laurito}


\bibliographystyle{plain}

%%%%%%%%%%%%%%%%%%%%%%%%%%%%%%%%%%%%%%%%%%%%%%%%%%%%%%%%%%%%%%%%%
% DOCUMENT
%%%%%%%%%%%%%%%%%%%%%%%%%%%%%%%%%%%%%%%%%%%%%%%%%%%%%%%%%%%%%%%%%

\begin{document}

\maketitle

\begin{abstract}
 Something great 
\begin{quotation}
 ``The best way to predict the future is to invent it - Alan Kay''
\end{quotation}

\end{abstract}

\section{Introduction}

In this paper we intend to create a computational model for programming languages 
running over the Software Defined Network (SDN) paradigm. For this purpose, we will 
define a graph grammar semantic, with it's own productions. This productions will 
represent actions that take place in our network, and we will see how this 
actions creates new behaviour in the model network. \\
The network will be model as a graph, and it's representation will be obtained by a YAML 
file, which will have to be delivered by the user of the programming language. We will explain
in further details how we manage to create a graph from a YAML file in the following section. \\ 
Finally, we will explain how this graph grammar can be used to model NEMO, a 
network oriented progamming language.

\section{Graph grammar production's}

We start by defining the most basic grammar production's. These 
productions are the one's that allows us to modify the network topoloy:


\begin{itemize}
  \item Node creation
  \begin{figure}[H]
    \[
       \xymatrix@C=25pt@R=16pt
       {
         {}
         \POS[]-<4.5pc,0pc>\drop{NEW\_NODE:}
         \POS[]+<0pc,.5pc> *+=<3.8pc,3.8pc>[F-]{} & {\hookleftarrow} &
         {}
         \POS[]+<0pc,.5pc> *+=<3.8pc,3.8pc>[F-]{} & {\hookrightarrow} &
         {\bullet} \ar@(ul,ur)^{idle}
         \POS[]+<0pc,.5pc> *+=<3.8pc,3.8pc>[F-]{1} &
       }
    \]
    \caption{Node creation}
    \protect\label{fig:nodecreation}
  \end{figure}
  This production creates a new logical node (this means that 
  the node could be either physic (which mean's it's a physical host with an 
  assocciated address) or logic (this mean's a group of physical nodes, meanning
  that the address could be a CIDR). \\
  It is important to remark that the created node is in an idle state, meaning 
  that is not interacting with any other node in the network. \\ 
  
  \item Service Creation
  \begin{figure}[H]
    \[
       \xymatrix@C=35pt@R=16pt
       {
        {}
         \POS[]-<5pc,0pc>\drop{NEW\_SERVICE:}
         \POS[]+<0pc,.5pc> *+=<3.8pc,3.8pc>[F-]{} & {\hookleftarrow} &
         {}
         \POS[]+<0pc,.5pc> *+=<3.8pc,3.8pc>[F-]{} & {\hookrightarrow} &
         {\bullet} \ar@(ul,ur)^{service} \ar@{->}^{dual}[r] \POS[]-<-0.5pc,.5pc> \drop{transmitter} 
         &
         {\bullet} \ar@(ul,ur)^{service} \POS[]-<-0.5pc,.5pc> \drop{receiver}
         \POS[]+<-2pc,.5pc> *+=<7.5pc,3.8pc>[F-]{} &
      }
    \]
    \caption{Service Creation}
    \protect\label{fig:servicecreation}
  \end{figure}
  In this production we define a new service. A service is either some property 
  implemented in nodes, or implemented in another node, wich will be used for generating 
  flows,this means that, when a flow is created, this flow will be of type service. The fact that the
  main idea is to send and receive information, is represented as dual nodes representing a 
  transmitter and a reicever in the graph.\\
  
  \item Node behaves as some endpiont of communication
  \begin{figure}[H]
    \[
       \xymatrix@C=35pt@R=35pt
       {
         {}\POS[]-<1pc,0pc>\drop{NODE\_IMPLEMENTS\_ROLE:}
         \\
         {\bullet}_{1} \ar@(ul,ur)^{idle} &
         {\bullet} \ar@(ul,ur)^{service} \ar@{->}^{dual}[d] \POS[]-<-0.5pc,.5pc> \drop{transmitter}
         & {\hookleftarrow} &
         {\bullet}_{1} \ar@(ul,ur)^{idle} &
         {\bullet} \ar@(ul,ur)^{service} \ar@{->}^{dual}[d] \POS[]-<-0.5pc,.5pc> \drop{transmitter}
         & {\hookrightarrow} &
         {\bullet}_{1} \ar@(ul,ur)^{idle}\ar@{->}^{behaves}[r]
         &
         {\bullet} \ar@(ul,ur)^{service} \ar@{->}^{dual}[d] \POS[]-<-0.5pc,.5pc> \drop{transmitter}
         \\
         & {\bullet} \ar@(r,d)^{service} \POS[]-<1.5pc,0pc> \drop{receiver}
         \POS[]+<-1pc,2pc> *+=<7.5pc,9pc>[F-]{} & & &
         {\bullet} \ar@(r,d)^{service} \POS[]-<1.5pc,0pc> \drop{receiver}
         \POS[]+<-1pc,2pc> *+=<7.5pc,9pc>[F-]{} & & &
         {\bullet} \ar@(r,d)^{service} \POS[]-<1.5pc,0pc> \drop{receiver}
         \POS[]+<-1pc,2pc> *+=<7.5pc,9pc>[F-]{}
       }
    \]
    \caption{Node implements role}
    \protect\label{fig:attachmentnodetoservice}
  \end{figure}
  This production is used to make node's to behave as one of the endpoints in 
  the communication of services. The node can only behave as one of the endpoint in 
  one service, and it's allow to implement more than one service.
  THIS RESTRICTION SHOULD BE DONE BY NEGATIVE APPLICATION\\
  
  \item Link creation
  \begin{figure}[H]
    \[
       \xymatrix@C=25pt@R=16pt
       {
        {\bullet}_{1}\ar@(ul,ur)^{idle} \POS[]-<3.8pc,0pc>\drop{NEW\_LINK:}
         \POS[]-<.4pc,-0.5pc> &
         {\bullet}_{2}\ar@(ul,ur)^{idle}
         \POS[]+<.4pc,0pc>
         \POS[]-<1pc,-0.5pc>*+=<7pc,3.8pc>[F-]{} & {\hookleftarrow} &
         {\bullet}_{1}\ar@(ul,ur)^{idle} 
         \POS[]-<.4pc,-0.5pc> &
         {\bullet}_{2}\ar@(ul,ur)^{idle}
         \POS[]+<.4pc,-0.5pc>
         \POS[]-<2pc,-0.5pc>*+=<7pc,3.8pc>[F-]{} & {\hookrightarrow} 
         \POS[]+<4.6pc,0.5pc> *+=<7pc,3.8pc>[]{} &
         {\bullet}_{1}\ar@(ul,ur)^{idle}\ar@{->}^{source}[d]
         \POS[]+<0pc,.5pc> &
         {\bullet}_{2} \ar@(ul,ur)^{idle}\ar@{->}^{target}[dl]
         \\
         & & & & & & {\bullet}_{link}\ar@(r,d)^{empty} \POS[]+<2pc,2pc>*+=<7pc,8pc>[F-]{} \POS[]-<-0.5pc,.5pc>
         \\
         %{\ar[d]} 
         \\
         {\bullet}_{1}\ar@(ul,ur)^{idle}\ar@{->}^{source}[d] \POS[]-<3.8pc,0pc>
         \POS[]-<.4pc,-0.5pc> &
         {\bullet}_{2}\ar@(ul,ur)^{idle}\ar@{->}^{target}[dl]
         \POS[]+<.4pc,0pc>
         \\
         {\bullet}_{link}\ar@(r,d)^{empty} \POS[]+<2pc,2pc>*+=<7pc,8pc>[F-]{} \POS[]-<-0.5pc,.5pc>
       }
    \]
    \caption{Link creation}
    \protect\label{fig:linkcreation}
  \end{figure}
  Here we define a new link between nodes 1 and 2. The only condition to apply 
  this production is that both nodes must be in an idle state and neither of them should have an existing 
  link created. This restriction is done by the negative condition showed in L.\\
  Note: Being in an idle state does not mean that they are doing nothing!. An 
  idle state behaves likes a wildcard, meaning that from this state, multiply actions 
  can be applied to the node.\\
  
  \item Link's property creation
  \begin{figure}[H]
    \[
       \xymatrix@C=35pt@R=16pt
       {
         {}\POS[]+<0pc,2pc>\drop{NEW\_LINK\_PROPERTY:}
         \\
         {\bullet}\ar@(ul,ur)^{empty} \POS[]-<0pc,.5pc>\drop{link}
         \POS[]+<0pc,.5pc> *+=<3.8pc,3.8pc>[F-]{} 
         & {\hookleftarrow} &
         {\bullet}\ar@(ul,ur)^{empty}\POS[]-<0pc,.5pc>\drop{link}
         \POS[]+<0pc,.5pc> *+=<3.8pc,3.8pc>[F-]{} & {\hookrightarrow} &
         {\bullet}\ar@(ul,ur)^{empty} \ar@{->}[r]\POS[]-<0pc,.5pc>\drop{link}
         \POS[]+<1.5pc,.5pc> *+=<6.5pc,3.8pc>[F-]{} &
         {\bullet}\ar@(ul,ur)^{property}
         \POS[]+<0pc,.5pc>*+=<4pc,4pc>[]{}\drop{}
       }
    \]
    \caption{Link's property creation}
    \protect\label{fig:linkpropertycreation}
  \end{figure}
  In this production we are allowing definition of link's property. This properties are going to be 
  perhaps physical properties in physical links (for example, we can define here bandwith, 
  average latency, perhpahs if it's either ethernet or wifi), and some other properties in logical 
  nodes.\\
  QUESTIONS:\\
  1) What properties could be assigned to a logical link? \\
  2) Can I always define properties in a node? (It really doesn't matter what is going on with 
  that link in the network?. Perphas it happens something similar as nodes, that we need some 
  'idle' state, for example, an empty state).\\
  
  \end{itemize}
  
  So far we have defined productions that which are intended to be use between the interactions 
  with the elements in the network. With the following productions, we will start using the productions
  listed before, in order to crate interaction between the elements in the network. \\
  
  \begin{itemize}
    \item Flow creation
  \begin{figure}[H]
    \[
       \xymatrix@C=35pt@R=35pt
       {
         {}\POS[]-<1pc,0pc>\drop{NEW\_FLOW:}
         \\
         {\bullet} \ar@(ul,ur)^{empty} \POS[]-<-0.5pc,.5pc> \drop{link} &
         {\bullet}_{1} \ar@(ul,ur)^{idle} \ar@{->}^{behaves}[r] \ar@{->}^{source}[l] &
         {\bullet} \ar@(ul,ur)^{service} \ar@{->}^{dual}[d] \POS[]-<-0.5pc,.5pc> \drop{transmitter}
         & {\hookleftarrow} &
         {\bullet}_{1} \ar@(ul,ur)^{idle} &
         {\bullet} \ar@(ul,ur)^{service} \ar@{->}^{dual}[d] \POS[]-<-0.5pc,.5pc> \drop{transmitter}
         & {\hookrightarrow} &
         {\bullet}_{1} \ar@(ul,ur)^{idle}\ar@{->}^{behaves}[r]
         &
         {\bullet} \ar@(ul,ur)^{service} \ar@{->}^{dual}[d] \POS[]-<-0.5pc,.5pc> \drop{transmitter}
         \\
         &
         {\bullet}_{2} \ar@(ul,ur)^{idle}\ar@{->}^{behaves}[r] \ar@{->}^{target}[ul] & 
         {\bullet} \ar@(r,d)^{service} \POS[]-<-1.5pc,0pc> \drop{receiver}
         \POS[]+<-1pc,2pc> *+=<7.5pc,9pc>[F-]{} & & &
         {\bullet} \ar@(r,d)^{service} \POS[]-<1.5pc,0pc> \drop{receiver}
         \POS[]+<-1pc,2pc> *+=<7.5pc,9pc>[F-]{} & & &
         {\bullet} \ar@(r,d)^{service} \POS[]-<1.5pc,0pc> \drop{receiver}
         \POS[]+<-1pc,2pc> *+=<7.5pc,9pc>[F-]{}
       }
    \]
    \caption{Flow creation}
    \protect\label{fig:flowcreation}
  \end{figure}
  In order to create a flow, I first need two things: 
  \begin{itemize}
    \item I need to have a link between the nodes
    \item I need to have in both nodes the service which will be the content type of the flow.  
   \end{itemize}
  
  Lets go through these rules one more time. The first rule is just telling us that, in order to be able to
  create a node, I need a connection between this node's (remember that this conection can be either
  logical or physical).\\
  The second rule is telling us that, in order to create a flow of type X (lets say for example, 
  that I want to create a ssh flow between this nodes), I need to guarantee that 
  both nodes support ssh connections. This is represented as both nodes having 
  a link to the service. \\
  Finally, it's important to notice that a flow is associated betwen nodes and a link 
  (this last one is going to be the place where it's flowing).\\
  \\
  THINS THAT HAD TO BE THINK A BIT MORE: \\
  1) Some rules may not be valid in this context ... ¿How do I deny multiple ssh 
  conections betwen same nodes? (Also, this sounds to me like a type condition, since another
  services may allow this behaviour, for example sql.) \\
  2) 
 
  \item Flow's properties creation
  \begin{figure}[H]
    \[
       \xymatrix@C=25pt@R=30pt
       {
         {}\POS[]-<0pc,0pc>\drop{NEW\_FLOW:}
         \\
         {\bullet}\ar@(ul,ur)^{active} \POS[]-<-.5pc,.5pc> \drop{my\_service} &
         {\bullet}_{1}\ar@(ul,ur)^{idle}\ar@{->}[l]\ar@{->}[r] &
         {\bullet}_{link}
         & {\hookleftarrow} & 
         {\bullet}\ar@(ul,ur)^{active} \POS[]-<-.5pc,.5pc> \drop{my\_service} &
         {\bullet}_{1}\ar@(ul,ur)^{idle}\ar@{->}[l]\ar@{->}[r] &
         {\bullet}_{link} 
         & {\hookrightarrow} & 
         {\bullet}\ar@(ul,ur)^{active} \POS[]-<-.5pc,.5pc> \drop{my\_service} &
         {\bullet}_{1}\ar@(ul,ur)^{idle}\ar@{->}[l]\ar@{->}[r]\ar@{->}[dr] &
         {\bullet}_{link}
         \\
         {\bullet}\ar@(ul,ur)^{active}\POS[]-<-.5pc,.5pc> \drop{my\_service} \POS[]+<3pc,2pc>*+=<10pc,8pc>[F-]{} &
         {\bullet}_{2}\ar@(ul,ur)^{idle}\ar@{->}[l]\ar@{->}[ur]
         & & &
         {\bullet}\ar@(ul,ur)^{active}\POS[]-<-.5pc,.5pc> \drop{my\_service} \POS[]+<3pc,2pc>*+=<10pc,8pc>[F-]{} &
         {\bullet}_{2}\ar@(ul,ur)^{idle}\ar@{->}[l]\ar@{->}[ur]
         & & &
        {\bullet}\ar@(ul,ur)^{active}\POS[]-<-.5pc,.5pc> \drop{my\_service} \POS[]+<3pc,2pc>*+=<10pc,8pc>[]{} &
         {\bullet}_{2}\ar@(ul,ur)^{idle}\ar@{->}[l]\ar@{->}[ur]\ar@{->}[r] &
         {\bullet}\ar@(ul,ur)^{flowing}\ar@{->}[u]\ar@{->}[d] \POS[]+<-3pc,1pc>*+=<10pc,11pc>[F-]{} 
         \POS[]-<-.5pc,.5pc>\drop{my\_flow\_service}
         \\\
         & & & & & & & & & &{\bullet}\POS[]-<.5pc,.5pc> \drop{my\_flows\_property}
      }
    \]
    \caption{Flow's property creation}
    \protect\label{fig:flowpropertycreation}
  \end{figure}
  This production allow's the creation of flow's properties.\\
  
  QUESTIONS:\\
  1) Can this properties be the actions associated in the nemo language ? 
  Perphaps properties can be understood as premises that have to be guaranteed 
  while the flow exist. \\
  2) Some of this properties could be dynamic (this has sense, since flow is the representation 
  of dynamism in the network). \\
    
\end{itemize} 

\section{Graph instantiation}

In this section we will focus in giving several examples of the usage of the 
grammar recently defined. The examples used in this section will be those defined in 
a previous document (REMEMBER TO REWRITE EXAMPLES HERE) \\ 
Lets first defined the first example, a network with two host and a router, 
which want to start a ssh connection. \\
There were two problems here:
\begin{itemize}
  \item Neither of the host had ssh active
  \item Neither of the host were connected to the router.
\end{itemize}

Let's model our network, and see how the grammar defined before help us to 
detect problems in what we are trying to do.\\
This would be the network represented in our model:
\begin{figure}[H]
    \[
       \xymatrix@C=25pt@R=30pt
       {
         {}\POS[]-<0pc,0pc>\drop{FIRST EXAMPLE}
         \\
         \\
         {\bullet}\ar@(ul,ur)^{idle} \POS[]-<-.5pc,.5pc> \drop{h1} &
         {\bullet}\ar@(ul,ur)^{idle} \POS[]-<-.5pc,.5pc> \drop{router} &
         {\bullet}\ar@(ul,ur)^{idle} \POS[]-<-.5pc,.5pc> \drop{h2}
      }
    \]
\end{figure}

Having model our network, what we can check is that NEW\_FLOW production can't be 
applied to my network (since there is no morphism between the left side of the production 
and this model). In this case, if I write a program for the situation described, 
we will have a compiler error, since there's going to be an instruction that 
cannot be done, and this is because the production associated with it cannot be 
executed.

\end{document}
